\documentclass{memoir}
%% Setup file for latex
%%%%%%%%%%%%%%%%%%%%%%%%%%%%%%%%%%%%%%%%%%%%%%%%%%
%% Load packages
\usepackage{mathptmx}  %% 420 pages          % for math fonts type 1
\usepackage[pdftex]{graphicx}           % for graphics files
\usepackage{floatflt}           % for ``floating boxes''
\usepackage{index}
\usepackage{relsize}            % for relative size fonts
\usepackage{amsmath}            % for amslatex stuff
\usepackage{amsfonts}           % for amsfonts
\usepackage{url}                % for \url,
\usepackage{listings}
\usepackage{booktabs}

%%\usepackage{subfigure}          % for subcaption command
%%%%%%%%%%%%%%%%%%%%%%%%%%%%%%%%%%%%%%%%%%%%%%%%%%
%%% The page


%%
%% Set up memor for book layout
%% paragraph

%% showtrims
%%\showtrimson
%%\trimXmarks


%%%%%%%%%%%%%%%%%%%%%%%%%%%%%%%%%%%%%%%%%%%%%%%%%%

%%%%%%%%%%
%% page layout
%% assume basic page size for now
\raggedbottom


%%%%%%%%%%
%% fonts
%% Review:
% Upright shape \textup{Upright shape} 
% Italic shape \textit{Italic shape} 
% Slanted shape \textsl{Slanted shape} 
% S MAL L CAP S S HAP E \textsc{Small Caps shape} 
% Series or weight 
% Medium series \textmd{Medium series} 
% Bold series \textbf{Bold series} 
% Family 
% Roman family \textrm{Roman family} 
% Sans serif family \textsf{Sans serif family} 
% Typewriter family \texttt{Typewriter family} 

%% which fonts?
%%\usepackage{avant}
%%\usepackage{palatcm}
\usepackage[T1]{fontenc}
%\usepackage[adobe-utopia]{mathdesign}
%\usepackage{aurical}

\renewcommand{\encodingdefault}{T1}
\usepackage[sc]{mathpazo}
\linespread{1.05}         % Palatino needs more leading (space between lines)

%% verbatim fonts
%\setverbatimfont{\fontencoding{T1}\fontfamily{cmss}\selectfont} 
%\setverbatimfont{\normalfont\ttfamily}





%%%%%%%%%%
%% titles

%%%%%%%%%%
%% divisions

%% chapter styles
\chapterstyle{ell}
%\chapterstyle{memman}
%\chapterstyle{veelo}
%\chapterstyle{bringhurst}
% \chapterstyle{crosshead}
% \chapterstyle{dowding}
% \chapterstyle{komalike}

% \chapterstyle{ntglike}
%\chapterstyle{tandh}
%\chapterstyle{wilsondob}

%%% pagestyle
\pagestyle{ruled}
%\pagestyle{companion}

% captions -- The class uses the following to specify the standard LaTeX caption style: 
% \captionnamefont{} 
% \captiontitlefont{} 
\captionstyle[\centering]{\raggedright} 
\captionwidth{\linewidth} 
% \normalcaptionwidth 
% \normalcaption 
\captiondelim{: } 
%\postcaption{\rule{\linewidth}{0.4pt}\par}

%%%%%%%%%%
%% pagination headers

%%%%%%%%%%
%% paragraphs, lists
\tightlists
%% WHAT TO DO HERE? \renewcommand{\paragraph}[1]{\marginpar{#1}}

%%%%%%%%%%
%% content lists

%%%%%%%%%%
%% floats and captions

%%%%%%%%%%
%% rows and columns

%%%%%%%%%%
%% page notes

%%%%%%%%%%
%% decorative text


%%%%%%%%%%
%% Boxes verbatims files

%%%%%%%%%%
%% cross referencing

%%%%%%%%%%
%% back matter





%%
%%
%%%%%%%%%%%%%%%%%%%%%%%%%%%%%%%%%%%%%%%%%%%%%%%%%%


%% http://article.gmane.org/gmane.comp.lang.r.general/137990
% \usepackage{listings}
% \lstset{%
% 	basicstyle=\scriptsize,
% 	breaklines=true,
% 	frame=single,
% 	literate=
% 		{<-}{$\leftarrow$}{2}
% }
% \renewcommand\lstlistlistingname{List of listings}

%%%%%%%%%%%%%%%%%%%%%%%%%%%%%%%%%%%%%%%%%%%%%%%%%%
%% Abbreviations (most are Rd-ish)
%% Flag something to look at -- XXX is easy to search for
\newcommand{\XXX}[1]{}%%{XXX-- #1 --XXX\\}


\newcommand{\R}{\textsf{R}}
\newcommand{\code}[1]{\texttt{#1}} % code
\newcommand{\qcode}[1]{\code{"#1"}} % quoted code 

\newcommand{\defn}[1]{\textit{#1}\index{\textit{#1}}}   % add in index
\newcommand{\command}[1]{\code{#1}} % name of command
\newcommand{\function}[1]{\code{#1}} % name of function
\newcommand{\constructor}[1]{\function{#1}\index{#1}}

\newcommand{\args}[1]{\code{#1}} % name of argument only
\newcommand{\argument}[2]{\args{#1}\index{#2|\texttt{#1}}} % name of an argument, plus
                                % function for index
\newcommand{\subcommand}[2]{\textit{#2} \args{#1}\index{#2|\code{#1}}} % name of an tk subcommand plus
                                % function for index
\newcommand{\subcommanda}[3]{\subcommand{#1}{#2} \textit{#3} }
\newcommand{\option}[2]{\args{#1}\index{#2|\code{#1}}} % name of an option plus constructor
\newcommand{\class}[1]{\code{#1}}  % a class
\newcommand{\generic}[1]{\code{#1}} % name of generic method -- no
\newcommand{\meth}[1]{\generic{#1}}     % single arg, no class
\newcommand{\method}[2]{\meth{#1}\index{#2|\code{#1}}} % name of method with
                                % class for index

\newcommand{\signal}[1]{\code{#1}} % name of signal
\newcommand{\dfn}[1]{\textit{#1}} % definition
\newcommand{\dfnref}[1]{\textit{#1}} % refer to a definition
\newcommand{\env}[1]{\texttt{#1}} % environment setting
\newcommand{\file}[1]{\texttt{#1}}
\newcommand{\kbd}[1]{\textmd{#1}}
\newcommand{\pkg}[1]{\texttt{#1}}
\newcommand{\opt}[1]{\texttt{#1}} % R option
\newcommand{\acronym}[1]{\texttt{#1}}

%% HTML
\newcommand{\tagger}[1]{\texttt{#1}}
\newcommand{\tagattr}[2]{\texttt{#1}} % #2 is the tag for index

\usepackage{fancyvrb}
\DefineShortVerb{\|}
\SaveVerb{ASSIGN}|<-|
%%\newcommand{\ASSIGN}{\code{\UseVerb{ASSIGN}}} % <- formats funny
\newcommand{\leftBracket}{$<$}
\newcommand{\rightBracket}{$>$}
%% THisis from the highlight pacakge
\newsavebox{\hlboxlessthan}%
\setbox\hlboxlessthan=\hbox{\verb.<-.}%
\newcommand{\ASSIGN}{{}{\usebox{\hlboxlessthan}}{}}
%%\newcommand{\ASSIGN}{\code{$<$-}}  %% <-
\newcommand{\backslashn}{\code{$\backslash$n}} %% \n

\newcommand{\GTK}{GTK+}
\newcommand{\TCL}{Tcl}
\newcommand{\Tcl}{\TCL}
\newcommand{\TK}{Tk}
\newcommand{\Tk}{Tk}
\newcommand{\tcltk}{Tcl/Tk}
\newcommand{\wxWidgets}{wxWidgets}
\newcommand{\Java}{Java}
\newcommand{\gWidgets}{gWidgets}

\newcommand{\TITLE}{Programming GUIs within R}
\title{\TITLE}
\newcommand{\PACKAGENAME}{ProgGUIInR}
\newcommand{\WINDOZE}{Windows}
\newcommand{\UNIX}{Unix}
\newcommand{\LINUX}{Linux}
\newcommand{\OSX}{Mac OS X}
%%
%%%%%%%%%%%%%%%%%%%%%%%%%%%%%%%%%%%%%%%%%%%%%%%%%%


\usepackage{color}
%%% Define some colors
\definecolor{gray70}{gray}{.70}
\definecolor{gray60}{gray}{.60}
\definecolor{gray50}{gray}{.50}
\definecolor{gray40}{gray}{.40}
\definecolor{gray25}{gray}{.25}

%% \usepackage{fancyhdr} % in memoir



\usepackage{fancyvrb}
%\usepackage{makeidx}
\usepackage{multicol}          % for making multiple columns
%%\usepackage{fancybox}           % for framing

\usepackage{prelim2e}           % put on bottom of each page
\renewcommand{\PrelimWords}{Draft version, do not circulate}

%\usepackage{/home/verzani/R/lib/R/share/texmf/Sweave}
%\usepackage{Sweave}

\usepackage{lscape}             % landscape tables
%% Save space things
%% http://www-h.eng.cam.ac.uk/help/tpl/textprocessing/squeeze.html 
%% \usepackage{float}        
% \usepackage{jvfloatstyle}       % redefine float.sty for my style. Hack
% \floatstyle{jvstyle}            
% \restylefloat{table}
% \restylefloat{figure}


\usepackage{natbib}
\bibpunct{(}{)}{;}{a}{,}{,}

\newcommand{\warning}[1]{Warning #1}
\newcommand{\aside}[1]{Aside #1}


%% An environment to flag an example to insert in index, allow for
%% referencing, and allow formatting
%% two arguments: title, label

%% XXX Fix me to make look nicer
\newcounter{Example}[chapter]
\def\theExample {\thechapter.\arabic{Example}}
\setcounter{Example}{1}

\newenvironment{example}[2]{
  \refstepcounter{Example} 
  \vspace{12pt}
  \noindent
  \textbf{Example \theExample: #1}
  \label{#2}
  \newline
}{
  %% how to finish
}


% \newenvironment{example}[2]{
%   \refstepcounter{Example} 
%   {\vspace{12pt}\noindent#1\label{#2}\hrulefill\newline}}%
%   {\noindent\hrulefill}

% \newenvironment{example}[2]{
%   \refstepcounter{Example} 
%   {\vspace{18pt}\noindent\large\sf Example \theExample: #1 \hfill}\label{#2}
% }{
%   \noindent{\rule{2in}{1pt}\vspace{12pt}}
% %% something at the finish too?
% }
%% Numbered code example
%% wrap around <<>>= @ environements to get number and labeling possibility
\newcounter{CodeExample}[chapter]
\def\theCodeExample {\thechapter.\arabic{CodeExample}}
\setcounter{CodeExample}{1}

\newenvironment{codeexample}[2]{
  \refstepcounter{CodeExample} 
  \noindent{Example \theCodeExample}\newline
  \hrulefill \newline}{
  {\noindent\hrulefill}
}



%%%%%%%%%%%%%%%%%%%%%%%%%%%%%%%%%%%%%%%%%%%%%%%%%%
%% Listings
%% for code listings. Make pretty?
\newenvironment{listing}{\begin{figure}}{\end{figure}}
\newcommand{\definition}[1]{#1}



%% Begin here
\usepackage{/Users/Verzani/R/R.framework/Resources/share/texmf/Sweave}
\begin{document}
\thispagestyle{empty}
\bibliographystyle{plainnat}



\chapter{The model-view-controller pattern}

%% orientation
The model-view-controller design pattern for GUIs (MVC) is a means to isolate the
data, the graphical represtation of the data, and the code that
connects the two. The data is stored in a \textit{model}, this data
may be represented by one or more \textit{views}, and a
\textit{controller} connects a model with a view. 


%% history
According to a wikipedia article
\begin{quotation}
  MVC was first described in 1979 by Trygve Reenskaug, then working
  on Smalltalk at Xerox PARC. The original implementation is described
  in depth in the influential paper ``Applications Programming in
  Smalltalk-80: How to use Model–View–Controller''.
\end{quotation}
It is widely implemented. In \pkg{RGtk2} some of the more complicated
widgets, such as the tree view, a text view etc., have an explicit
specification of the model. In \pkg{tcltk}, the TCL variables play the
role the model and the TK widgets are views of the model. 

In \R\/ an implementation is given in the bioconductor pacakge
\pkg{MVCClass} using S4 classes. In a subsequent chapter, we provide
a lightweight implementation using the \pkg{proto} package.

%% responsibilities
A common use, found in the \pkg{ggobi} package say, is graphical
brushing. That is, when a user identfies certain points in one graph
displaying a data set, the same points are highlighted in a different
graph of the same data set. 

In MVC language, we have the model is the data set, and there are two
views -- the two graphs. When one graph is brushed a controller
informs the model that there is a change in selection. The model then
notifies any view that the selection has changed (again through a
controller) and the view updates its representation accordingly. By
inserting a controller between the model and the view, the two are
decoupled which provides several benefits. These benefits include, the
overall program flow is easier to debug, the decoupling allows views
and models to be reused, and the model can be changed independent of the
view (so one can update the graphical displays without having to
remake the graphs). Of course, the benefits come at a cost --
increased complexity, atleast conceptually.


\section{A basic implementation}
\label{sec:basic-implementation}
To illustrate the concept and the responsibilities for each part we
present an implementation using the \pkg{proto} package. This package
extends \R's environments to create a somewhat object-oriented
programming style. 

Our implementation follows somewhat that given in the \pkg{pygtkmvc}
python package
(\url{http://sourceforge.net/apps/trac/pygtkmvc/wiki}). There are many
different implementations, this one is relatively straightforward.


\subsection{A base trait}
\label{sec:base-trait}

The \pkg{proto} package implement protoype programming which is not
technically object-oriented, but does allow for the main features:
a concept of object properties, object methods, and object
inheritance. However, the concept of a class is not used. (The
\pkg{mutatr} package does something similar.) 

Instead of classes, one defines a ``trait'' that provides the standard
properties and methods for an instance. Sub traits can inherits these
and modify them, as we will illustrate. 

We load two package to start.
\begin{Schunk}
\begin{Sinput}
 require(proto)
 require(digest)
\end{Sinput}
\end{Schunk}

We begin by defining a base trait from which our \code{Model},
\code{View} and \code{Controller} traits will inherit. The naming
convention is uppercase camel case.
\begin{Schunk}
\begin{Sinput}
 BaseTrait <- proto()
\end{Sinput}
\end{Schunk}
The \pkg{proto} package allows for properties and methods to be
defined within the \code{proto} call, but for typesetting reasons we
will assign using the \code{\$} notation for environments. 

We define a class property to keep track of the type of \code{proto}
object we have, as there is not built in class concept.
\begin{Schunk}
\begin{Sinput}
 BaseTrait$class <- "Base"
\end{Sinput}
\end{Schunk}

Defining a property, as above, is straightforward. Defining a method
is a bit different, as the functions have an initial argument of
\code{.}. This allows the object to be passed to the function body. As
\code{proto} objects are just environments, they are mutatable, so
when the assignment is made in this method definition, the value of
\code{class} property is updated in the object outside the function body.
\begin{Schunk}
\begin{Sinput}
 BaseTrait$add_class <- function(., newclass) .$class <- c(newclass, .$class)
\end{Sinput}
\end{Schunk}

We define a simple method to append a value to a property which is
storing a list of values.
\begin{Schunk}
\begin{Sinput}
 BaseTrait$append <- function(., name, value, key) {
   val <- get(name, envir=.)
   if(is.list(val)) {
     if(!missing(key))
       val[[key]] <- value
     else
       val[[length(val)+1]] <- value
   } else {
     val <- c(val, value)
   }
   assign(name, val, envir=.)
 }
\end{Sinput}
\end{Schunk}

The \pkg{proto} package allows for introspection -- determining
properties at run time. We implement some methods for doing so. First,
an implementation mirroring the \code{is} function of S4 programming.
\begin{Schunk}
\begin{Sinput}
 BaseTrait$is <- function(., class=NULL) {
   if(!is.null(class))
     class %in% .$class
   else
     TRUE
 }
\end{Sinput}
\end{Schunk}
This function returns all properties and methods. The trick is that
children of a \code{proto} object inherit methods and properties, but
\code{ls} does not list those, so we walk backwards using
\code{parent.env} (called in a OO manner).
\begin{Schunk}
\begin{Sinput}
 BaseTrait$list_objects <- function(., class=NULL) {
   s <- .
   if(!s$is(class))
     return(list())
   out <- ls(s, all.names=TRUE)
   while(is.proto(s <- s$parent.env())) {
     if(s$is(class))
       out <- c(out, ls(s, all.names=TRUE))
   }
   unique(out)
 }
\end{Sinput}
\end{Schunk}

This method returns all methods of a certain class, as defined by the
class property (not the class in the S3 or S4 sense).
\begin{Schunk}
\begin{Sinput}
 BaseTrait$list_by_class <- function(., class) {
   out <- .$list_objects()
   out <- sapply(out, function(i) {
     obj <- get(i, envir=.)
     if(is.proto(obj) && obj$is(class))
       obj
   })
   out[!sapply(out, is.null)]
 }
\end{Sinput}
\end{Schunk}
Next two methods to list just the properties and the methods.
\begin{Schunk}
\begin{Sinput}
 BaseTrait$list_properties <- function(., return_names=FALSE, class=NULL) {
   ## will return objects or just names if return_names=TRUE
   ## can pass class= value if desired
   out <- .$list_objects(class=class)
   out <- sapply(out, function(i) {
     ## skip "class" and dot names
     if(i != "class" && !grepl("^\\.",i) && !grepl("^\\.doc_",i)) {
       obj <- get(i, envir=.)
       if(!is.function(obj))
         obj
     } else {
       NULL
     }
   })
   out <- out[!sapply(out, is.null)]
   if(return_names)
     names(out)
   else
     out
 }
\end{Sinput}
\end{Schunk}
\begin{Schunk}
\begin{Sinput}
 BaseTrait$list_methods <- function(.) {
   nms <- .$list_objects()
   ind <- sapply(nms, function(i) {
     is.function(get(i,envir=.))
   })
   nms[ind]
 }
\end{Sinput}
\end{Schunk}

Finally, we make a convenience function to call a method provided that
method exists and is a function.

\begin{Schunk}
\begin{Sinput}
 BaseTrait$do_call <- function(., fun, lst=list()) {
   if(exists(fun, envir=.) && is.function(FUN <- get(fun, envir=.))) 
     do.call(FUN, c(., lst))
 }
\end{Sinput}
\end{Schunk}

%% Model
\subsection{A model trait}
\label{sec:model-trait}
A model consists of properties and methods to manipulate these
properties. In addition we implement the observer pattern. An observer
is a controller. When a model property changes, all the observers are
notified of this. The observer can then update any views it is
associated with. In order to implement this, we must change the
property values through the \code{setattr} method. For convenience,
the \code{init} method will create \code{get}/\code{set} pairs for
interacting with the property values by name. 

\begin{Schunk}
\begin{Sinput}
 Model <- BaseTrait$proto()
 Model$add_class("Model")
\end{Sinput}
\end{Schunk}




We implement the observer pattern by defining a few key methods. An
oberver is a controller instance (defined later). The following just
stores these in a list using a private property.
\begin{Schunk}
\begin{Sinput}
 Model$.observers = list()               # private property (leading .)
 Model$add_observer <- function(., observer) {
   if(is.proto(observer) && observer$is("Controller")) {
     id <- length(.$.observers) + 1
     .$.observers[[id]] <- observer
   }
 }
\end{Sinput}
\end{Schunk}
The \pkg{proto} package provides the \code{identical} method to
compare two proto objects. We use this to remove an observer when requested.
\begin{Schunk}
\begin{Sinput}
 Model$remove_observer <- function(., observer) {
   if(!missing(observer) && (is.proto(observer) && observer$is("Controller"))) {
     ind <- sapply(.$.observers, function(i) i$identical(observer))
     if(any(ind))
       sapply(which(ind), function(i) .$.observers[[i]] <- NULL)
   }
 }
\end{Sinput}
\end{Schunk}
This is the key method, which is called when a property value is
changed through \code{setattr}. The controllers use a naming
convention. If a property \code{prop1} is changed, then the methods
\code{property\_prop1\_value\_changed} and \code{model\_value\_changed}, if
present in the controller, are called.
\begin{Schunk}
\begin{Sinput}
 Model$notify_observers <- function(., key=NULL, value=NA, old_value=NA) {
   sapply(.$.observers, function(i) {
     if(digest(value) != digest(old_value)) { #serialize, then compare
       if(!is.null(key)) {
         i$do_call(sprintf("property_%s_value_changed",key),
                   list(value=value, old_value=old_value))
       }
       i$do_call("model_value_changed", list()) # always call if present
     }
   })
   invisible()
 }
\end{Sinput}
\end{Schunk}

The methods \code{getattr} and \code{setattr} are used to interact
with the model's properties,
\begin{Schunk}
\begin{Sinput}
 Model$getattr = function(., key) get(key, envir=.)
 Model$setattr = function(., key, value) {
   old <- .$getattr(key)
   assign(key, value, envir=.)
   .$notify_observers(key=key, value=value, old_value=old)
 }
\end{Sinput}
\end{Schunk}

On initialization, a model has get/set methods defined for its
properties, as a convenience to using \code{getattr} and \code{setattr}.
\begin{Schunk}
\begin{Sinput}
 Model$init = function(.) {
   sapply(.$list_properties(return_names=TRUE), function(i) {
     assign(paste("get_",i, sep=""),
            function(.,...) .$getattr(i),
            envir=.)
     assign(paste("set_", i, sep=""),
            function(., value, ...) .$setattr(i,value),
            envir=.)
   })
   invisible()
 }
\end{Sinput}
\end{Schunk}

%% view
\subsection{A view trait}
\label{sec:view-trait}

A view typically provides a visual representation of a model property or
properties. (Not all case, as a model could also be a view, etc..) Our
view trait is oriented around using \pkg{gWidgets} to provide the
graphical widgets.

\begin{Schunk}
\begin{Sinput}
 View <- BaseTrait$proto()
 View$add_class("View")
\end{Sinput}
\end{Schunk}

The basic view properties include a list of attributes to pass to the
widget constructor (\code{make\_ui}) and a list of widgets, to which
we provide a few convenience methods.
\begin{Schunk}
\begin{Sinput}
 View$attr <- list()                   # passed to widget constructor
 View$widgets <- list()                     # lists all widgets in the view
 View$get_widgets <- function(.) .$widgets
 View$get_widget_by_name <- function(., key) .$get_widgets()[[key]]
\end{Sinput}
\end{Schunk}
The user interface for a view is created by the \code{make\_ui}
method. This should create the widgets and save those that will be
referenced later in the \code{widgets} property.
\begin{Schunk}
\begin{Sinput}
 View$make_ui <- function(., cont, attr=.$attr) {}
\end{Sinput}
\end{Schunk}
The view has two distinct states -- before the widget is realized and after. It
is important to be able to determine which state the widget is in.
\begin{Schunk}
\begin{Sinput}
 View$is_realized <- function(.) 
   length(.$get_widgets()) && isExtant(.$get_widgets()[[1]])
\end{Sinput}
\end{Schunk}

Communication between the model and view is done through the
controller. These methods are there to provide a standard interface in
the simplest cases.  These just provide a convenient means for the
controller, they do not sychronize with the model, as the view does
not know the model or even the controller. 

\begin{Schunk}
\begin{Sinput}
 get_value_from_view = function(.) {}
 set_value_in_view = function(., widget_name, value) {
   if(.$is_realized()) {
     widget <- .$get_widget_by_name(widget_name)
     svalue(widget) <- value
   }
 }
\end{Sinput}
\end{Schunk}

Finally, we define similar functions to hide or disable the view.
\begin{Schunk}
\begin{Sinput}
 View$enabled <- function(., bool) 
   if(.$is_realized())
   invisible(sapply(.$get_widgets(), function(i) enabled(i) <- bool))
 View$visible <- function(., bool) 
   if(.$is_realized())
   invisible(sapply(.$get_widgets(), function(i) visible(i) <- bool))
\end{Sinput}
\end{Schunk}

%% controller
\subsection{A controller trait}
\label{sec:controller-trait}
Controllers have the difficult task of implementing the core logic
that connects the various views and models. A controller needs to know
the model and the view and provide a means for the two to communicate
back and forth, if desired. 
\begin{Schunk}
\begin{Sinput}
 Controller <- BaseTrait$proto()
 Controller$add_class("Controller")
\end{Sinput}
\end{Schunk}
We first define a \code{model} property and some methods to interact
with it. When setting the model, we also take care to update the
observers including the \code{adapters} which are a simple form of a
controller discussed later.
\begin{Schunk}
\begin{Sinput}
 Controller$model <- NULL
 Controller$get_model <- function(.) .$model
 Controller$set_model <- function(., model) {
   if(is.proto(model) && model$is("Model")) {
     .$model$remove_observer(.)
     .$model <- model
     .$model$add_observer(.)
     sapply(.$.adapters, function(i) i$set_model(model))
   }
 }
\end{Sinput}
\end{Schunk}

Similarly, we define a \code{view} property and some methods to
interact with it.
\begin{Schunk}
\begin{Sinput}
 Controller$view <- NULL
 Controller$get_view <- function(.) .$view
 Controller$set_view <- function(., view) {
   if(is.proto(view) && view$is("View")) {
     if(is.proto(.$get_view()) && .$get_view()$is("View"))
       .$remove_view()
     .$view <- view
     sapply(.$.adapters, function(i) i$set_view(view))
   }
 }
 Controller$remove_view <- function(.) 
   sapply(.$.adapters, function(i) i$remove_view())
\end{Sinput}
\end{Schunk}
These methods are used in the definition of the adapter pattern given
later. They are used to synchronize changes in the model with the view
and vice versa.
\begin{Schunk}
\begin{Sinput}
 Controller$update_from_model <- function(.) {}
 Controller$update_from_view <- function(.) {}
\end{Sinput}
\end{Schunk}
This initialization method is used to connect the controller to the
model (as an observer) and to propogate the model values to the view
the initial time.
\begin{Schunk}
\begin{Sinput}
 Controller$init <- function(.) {
   if(!is.null(.$get_model())) {
     .$update_from_model()
     .$get_model()$init()
     .$get_model()$add_observer(.)
   }
   if(!is.null(.$get_view())) {
     .$update_from_view()
   }
   .$register_adapters()
   ## call value_changed methods to update any views
   nms <- .$list_methods()
   sapply(nms[grep("property_(.*)_value_changed$", nms)],
          function(i) {
            prop <-  gsub("property_(.*)_value_changed$","\\1",i)
            get(i, envir=.)(., .$get_model()$getattr(prop),NA)
          })
   invisible()
 }
\end{Sinput}
\end{Schunk}
The controller is used as an observer for its model. Sub classes may
override methods such as these to implement specific actions when the
model is changed. These follow the naming convention needed by our
implementation of the observer pattern.
\begin{Schunk}
\begin{Sinput}
 Controller$model_value_changed <- function(.) {}
\end{Sinput}
\end{Schunk}
\begin{Schunk}
\begin{Sinput}
 Controller$property_PROPERTYNAME_value_changed <- function(., value, old_value) {}
\end{Sinput}
\end{Schunk}

Defining a controller can be a bit involved. The adapter pattern
simplifies this for the simple case that a single property is being
observed and the view has just a single widget to update.

We define an adapter using a list of lists:
\begin{Schunk}
\begin{Sinput}
 ## list of adapters. Each adapter specified with a list. E.g.,
 ## list(property="propname",
 ##      view_name="viewname",
 ##      add_handler_name=c("addHandlerChanged"), # or NULL to suppress
 ##      handler_user_data=NULL
 ##      )
 Controller$adapters <- list()
\end{Sinput}
\end{Schunk}
The adapters are constructed by the \code{register\_adapters} method
which is called in the \code{init} method. The actual adapter
instances are stored in this private property.
\begin{Schunk}
\begin{Sinput}
 Controller$.adapters <- list()
 Controller$.handlerIDs <- list()
\end{Sinput}
\end{Schunk}
Finally, our method to register the adapters is defined using the
\code{Adapter} trait given below.
\begin{Schunk}
\begin{Sinput}
 Controller$register_adapters <- function(.) {
   if(length(.$adapters) && !length(.$.adapters)) {
     .$.adapters <- lapply(.$adapters, function(i) {
       Adapter$proto(model=.$get_model(),
                     view=.$get_view(),
                     property=i$property,
                     view_widget_name=i$view_widget_name,
                     add_handler_name=i$add_handler_name,
                     handler_user_data=i$handler_user_data
                     )
     })
   }
   if(length(.$.adapters))
                                    sapply(.$.adapters, function(i) i$init())
 }
\end{Sinput}
\end{Schunk}


Next we define the adapter trait.
\begin{Schunk}
\begin{Sinput}
 Adapter <- Controller$proto()
 Adapter$add_class("Adapter")
\end{Sinput}
\end{Schunk}
We define some properties of the adapter. We specify the model
property and name of the widget in the view for starters.
\begin{Schunk}
\begin{Sinput}
 Adapter$property <- NULL
 Adapter$view_widget_name <- NULL       # otherwise last one
\end{Sinput}
\end{Schunk}
The view communicates back to the model through the controller through
a callback. This defines the \pkg{gWidgets} ``\code{addHandlerXXX}'' to be used. One
can leave this an empty string for no interaction
\begin{Schunk}
\begin{Sinput}
 Adapter$add_handler_name <- c("addHandlerChanged") # 1 or more
 Adapter$handler_user_data=NULL
\end{Sinput}
\end{Schunk}

This method is called by \code{init} to add a moel observer that
updates the widget value.
\begin{Schunk}
\begin{Sinput}
 Adapter$update_from_model = function(.) {
   ## set up model to notify view For example:
   view <- .$get_view()
   meth_name<- sprintf("property_%s_value_changed", .$property)
   assign(meth_name,
          function(., value, old_value) {
            view$set_value_in_view(.$view_widget_name, value)
          },
          envir = .)
   ## call method
   get(meth_name, envir=.)(., .$get_model()$getattr(.$property), NA)
 }
\end{Sinput}
\end{Schunk}
This method is called by \code{init} to add handlers to the widget to
propogate changes back to the model.
\begin{Schunk}
\begin{Sinput}
 Adapter$update_from_view <- function(.) {
   ## here view knows about model through controller (this adapter)
   if(!.$get_view()$is_realized()) return()
   if(!is.null(.$view_widget_name))
     widget <- .$get_view()$get_widget_by_name(.$view_widget_name)
   else
     widget <- tail(.$get_view()$get_widgets(), n=1)[[1]]
   ## gWidgets specific call to set up control between model and
   ## view
   if(is.null(.$add_handler_name))
     .$add_handler_name="addHandlerChanged"
   sapply(.$add_handler_name, function(i) {
     if(i != "") {
       lst <- list(obj=widget,
                   handler=function(h,...) {
                     . <- h$action$adapter
                     
                     ## set property in model using name
                     value <- svalue(h$obj)
                     if(isExtant(h$obj)) {
                       .$model$setattr(.$property, value)
                     }
                   },
                   action=list(adapter=.))
       .$append(".handlerIDs", do.call(i, lst))
     }
   })
 }
\end{Sinput}
\end{Schunk}
This method is called to remove a view, disconnecting the handlers
that have been defined first.
\begin{Schunk}
\begin{Sinput}
 Adapter$remove_view <- function(.) {
   if(exists(".handlerIDs", .))
     sapply(.$.handlerIDs, function() removeHandler(.$get_view(), i))
 }
\end{Sinput}
\end{Schunk}
This initialization method sets up the adapter.
\begin{Schunk}
\begin{Sinput}
 Adapter$init <- function(.) {
   ## check that we are all there
   if(!is.null(.$property) &&
      (is.proto(model <- .$get_model()) && model$is("Model")) &&
      (is.proto(view <- .$get_view()) && view$is("View"))) {
     .$update_from_model()
     .$update_from_view()
   } else {
     warning("Adapter does not have view, model and property")
   }
   .$model$add_observer(.)
 }
\end{Sinput}
\end{Schunk}


\subsection{Examples}
\label{sec:examples}

\paragraph{Silly example using an adapter}
Our first example uses a model with just two properties.
\begin{Schunk}
\begin{Sinput}
 model <- Model$proto(prop1=1, prop2="button label")
\end{Sinput}
\end{Schunk}
We still need to initialize this model if the get/set pairs are desired.

Our view, for sake of illustration, has a text area and a button.
\begin{Schunk}
\begin{Sinput}
 require(gWidgets)
 options(guiToolkit="RGtk2")
 view <- View$proto(make_ui=function(., cont, attr=.attr) {
   .$widgets[['toplevel']] <- w <- gwindow("Example")
   g <- ggroup(cont=w, horizontal=FALSE)
   .$widgets[["text"]] <- gtext("", cont=g)
   .$widgets[["button"]] <- gbutton("button", cont = g)
 })
\end{Sinput}
\end{Schunk}

We use an adapter to connect the text widget with the first property
and the button widget with the second. As we don't modify the model
when the button is clicked, we specify the handler name with an empty string.
\begin{Schunk}
\begin{Sinput}
 adapter <- Controller$proto(model=model, view=view,
                               ## call and adapter
                               adapters=list(
                                 prop=list(
                                   property="prop1",
                                   view_widget_name="text"
                                   ),
                                 ## button doesn't need to set model
                                 button=list(property="prop2",
                                   view_widget_name="button",
                                   add_handler_name="")
                                 )
                             )
 model$init()
 view$make_ui()
 adapter$init()
\end{Sinput}
\end{Schunk}


To do the same thing with a controller is a bit more involved. We woul override the methods
\code{update\_from\_model} and \code{update\_from\_view} which set up
the communication between the model and the view.
\begin{Schunk}
\begin{Sinput}
 controller <- Controller$proto(model=model, view=view)
\end{Sinput}
\end{Schunk}
\begin{Schunk}
\begin{Sinput}
 controller$update_from_model <- function(.) {
   .$property_prop1_value_changed <- function(., value, old_value) 
     svalue(.$view$widgets[['text']]) <- value
   .$property_prop2_value_changed <- function(., value, old_value)  {
     button <- .$view$get_widget_by_name("button")
     svalue(button) <- value
   }
 }
\end{Sinput}
\end{Schunk}

This defines communication between the text entry and the model. No
action is given to the button, although in practice this wouldn't make
any sense.
\begin{Schunk}
\begin{Sinput}
 controller$update_from_view <- function(.) {
   widget <- .$get_view()$get_widget_by_name("text")
   .$append(".handlerIDs", addHandlerChanged(widget, handler = function(h,...) {
     . <- h$action$controller
     model <- .$get_model()
     val <- svalue(h$obj)
     model$set_prop1(val)
   }, action=list(controller=.))
            )
 }
\end{Sinput}
\end{Schunk}

\paragraph{One model property two views}
Our next example shows how we can share a model among views, in this
case to sychronize a spinbutton and a slider widget.

\begin{Schunk}
\begin{Sinput}
 model <- Model$proto(value=1)
 model$init()
\end{Sinput}
\end{Schunk}
\begin{Schunk}
\begin{Sinput}
 view <- View$proto()
 view$make_ui <- function(.) {
   .$widgets[['toplevel']] <- (w <- gwindow("Example"))
   g <- ggroup(cont = w, horizontal=TRUE, expand=TRUE)
   .$widgets[['slider']] <- gslider(from=0, to=10, by=1, cont=g, expand=TRUE)
   .$widgets[['spinner']] <- gspinbutton(from=0, to=10, by=1, cont=g)
 }
 view$make_ui()
\end{Sinput}
\end{Schunk}

\begin{Schunk}
\begin{Sinput}
 controller <- Controller$proto(model=model, view=view)
\end{Sinput}
\end{Schunk}
We bypass \code{update\_from\_model} and define the value changed
method directly:
\begin{Schunk}
\begin{Sinput}
 controller$property_value_value_changed <- function(., value, old_value) {
   ## update both widget
   widgets <- lapply(c("slider","spinner"), function(i) 
                     .$view$get_widget_by_name(i))
   sapply(widgets, function(i) svalue(i) <- value)
 }
\end{Sinput}
\end{Schunk}
The same handler is used for each widget to update the model, giving
us the following code to add the callbacks when the GUI is updated by
the user.
\begin{Schunk}
\begin{Sinput}
 controller$update_from_view <- function(.) {
   handler <- function(h,...) {
     . <- h$action$controller
     model <- .$get_model()
     val <- svalue(h$obj)
     model$set_value(val)
   }
   widgets <- lapply(c("slider","spinner"), function(i) 
                     .$view$get_widget_by_name(i))
   IDs <- sapply(widgets, function(i) addHandlerChanged(i, handler=handler, action=list(controller=.)))
 }
 controller$init()
\end{Sinput}
\end{Schunk}

                                      
                              





\end{document}

%%% Local Variables: 
%%% mode: latex
%%% TeX-master: t
%%% End: 
