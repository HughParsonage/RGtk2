


\section*{About this book}
\label{sec:about}


% ML: this is repeated below, do we really want to lead with this?
\R\/ has a number of packages that provide a link between the \R\/
user and a graphical toolkit, such as \code{tcltk}, \code{RGtk2} and
\code{qtbase}. In addition, an \R\/ user can interface with
Java, Python or other external languages to provide access to graphical
toolkits within those languages. This book is about writing graphical
user interfaces (GUI) within \R\/ that do not rely on knowing an
external programming language.  

The R language, like its predecessor S, is designed for interactive
use through a command line interface (CLI). However, the graphical
user interface (GUI) has emerged as an effective alternative,
depending on the specific task and the target user base. Currently,
there is a range of graphical interfaces for \R\/ that are programmed
within R. For example, several package authors have provided GUIs for
their functions. Examples include \pkg{limmaGUI}, \pkg{caGUI},
\pkg{clustTool}, \pkg{Metabonomic}, and others. There are a
few tools to automatically generate such GUIs, such as the \pkg{fgui}
package and the \function{guiDlgFunction} function from the
\pkg{svDialogs} package. Other authors have provided graphical
interfaces to explore data sets, such as \pkg{ggobi}, \pkg{playwith},
\pkg{latticist} and \pkg{aplpack}. Still others have provided packages
with GUIs aimed at allowing students to perform some simulation, e.g.,
\pkg{teachingDemos}. The \pkg{rattle} package provides an interface
for several data mining operations. The \pkg{Rcmdr} package provides a
menu- and dialog-driven interface to a wide range of \R's
functionality. There are several user-contributed plugins that extend
the \code{Rcmdr}. Additionally, as \R\/ finds wider usage outside of
academia, it is not uncommon for people who work in a team setting to
desire an interface to their \R\/ code that allows non-\R\/ users
access.

Most all of these examples are within the scope of this book. We set out to
show that for many purposes adding a graphical interface to one's work
is not terribly sophisticated nor time-consuming.  This book does not
attempt to cover the development of GUIs that require knowledge of
another programming language, although several such projects
exist. Many of these are general front-ends to \R\/, such as the
Java-based GUI \pkg{JGR}, the \pkg{rkWard} GUI for KDE, the
\pkg{biocep} GUI written using Java and the \pkg{RServe} package, or
even the Windows GUI that comes with \R's Windows package. There are
also several special purpose GUIs, like \pkg{iPlots}, which are
largely implemented in Java, relying on \pkg{rJava}, a native
interface between \R\/ and Java.

% ML: It seems like rJava-based GUIs like CADStat and Deducer DO fall into
% our scope, but I would hate to have to document Swing...
%x JV: Me too! It may be artificial, but rJava does require one to write Java code
% to do many things. For example, a simple callback, so it is in my mind
% more on the java side of things than the R side.

The bulk of this text covers four different packages for writing GUIs
in \R. The \pkg{gWidgets} package is covered first. This provides a
common programming interface over several \R\/ packages that implement
low-level, native interfaces to GUI toolkits. The \pkg{gWidgets}
interface is much simpler -- and less powerful -- than the native
toolkits, so is useful for a programmer who does not wish to invest
too much time into perfecting a GUI. There are a few other packages
that provide a high-level \R\/ interface to a toolkit such as
\code{rpanel} or \code{svDialogs}, but we focus on this one.

% ML: but gWidgets is much more general than either of those, right?
% JV: Yes

The next three parts introduce the native interfaces upon which
\pkg{gWidgets} is built. These offer fuller and more direct control of
the underlying toolkit and thus are well suited the development of GUIs
that require special features or performance characteristics.  The
first of these is the \pkg{RGtk2} package which provides a link
between \R\/ and the cross-platform \GTK\/ library. \GTK\/ is mature,
feature rich and leveraged by several widely used
projects. 

Another mature and feature-rich toolkit is \Qt, an open-source C++ library
from Nokia. The \R\/ package \pkg{qtbase} provides a native interface
from \R\/ to \Qt.  As \Qt\/ is implemented in C++, it is designed
around the ability to create classes that extend the \Qt\/
classes. \pkg{qtbase} supports this from within \R\/, although such
object oriented concepts may be unfamiliar to many \R\/ users.

Finally, we discuss the \pkg{tcltk} package, which provides the \R\/
user access to the \TK\/ libraries. Although not as modern as \GTK\/
nor \Qt, these libraries come pre-installed with the Windows binary,
thus avoiding installation issues for the average end-user. The
bindings to \TK\/ were the first ones to appear for \R\/ and 
several of the GUI projects above, notably \pkg{Rcmdr}, use this
toolkit.

These four main parts are preceeded by an introductory chapter on GUIs
and followed by a chapter on web GUIs.

The text is written with the belief that much can be learned by
studying examples, and so several examples are given. Some of these
are meant as sketches of what can be done, while a few illustrate how
to code actual useful interfaces.  This text can't expect to cover all
of the features of a graphical toolkit. For the \pkg{tcltk},
\pkg{RGtk2} and \pkg{qtbase} packages, the underlying toolkits have
well documented APIs.


This text comes with an accompanying package \pkg{\PACKAGENAME}. This
package includes the complete code for all the examples. In order to
save space, some examples in the text have code that is not shown. The
package provides the functions \code{browsegWidgetsFiles},
\code{browseRGtk2Files}, \code{browseQtFiles} and
\code{browseTclTkFiles} for browsing the examples from the respective
chapters. Additionally, this package will contain vignettes describing
aspects that did not make it into the text.


This text was written with the \pkg{Sweave} package. To suppress
superflous output an assignment to a variable named \code{QT} is made
at times.
 


