


\section*{About this book}
\label{sec:about}

Two common types of user interfaces in statistical computing are the
command line interface (CLI) and the graphical user interface
(GUI). The usual CLI consists of a textual console where the user
types a sequence of commands at a prompt and the output of the
commands is printed to the console as text. The \proglang{R} console
is an example of a CLI. A GUI is the primary means of interacting with
desktop environments, like Windows and Mac OS, and statistical
software like JMP. GUIs are contained within windows, and
resources, such as documents, are represented by graphical icons.
User controls are packed into hierarchical drop-down menus, buttons,
sliders, etc. The user manipulates the windows, icons and menus with a
pointer device, such as a mouse.

The R language, like its predecessor S, is designed for interactive
use through a command line interface (CLI), and the CLI remains the
primary interface to R. However, the graphical user interface (GUI)
has emerged as an effective alternative, depending on the specific
task and the target audience.  With respect to GUIs, we see R users
falling into three main target audiences: those who are familiar with
programming \proglang{R}, those who are still learning how to program,
and those who have no interest in programming.

On some platforms, such as Windows and Mac OS, \R{} has graphical
front-ends that provide a CLI through a text console control. Similar
examples include the web-based RStudio\texttrademark{} IDE, the
Java-based JGR and the RKWard GUI for the Linux KDE desktop. Although
these interfaces are GUIs, they are still very much in essence CLIs,
in that the primary mode of interacting with \proglang{R} is the
same. Thus, these GUIs appeal mostly to those who are comfortable with
\proglang{R} programming.

A separate set of GUIs target the second group of users, those
learning the \proglang{R} language. Since this group includes many
students, these GUIs are often designed to teach general statistical
concepts in addition to \proglang{R}.  A CLI component is usually
present in the interface, though it is deemphasized by the surrounding
GUI, which is analogous to a set of ``training wheels'' on a
bicycle. An example of such a GUI is \proglang{R} Commander, which
provides a menu- and dialog-driven interface to a wide range of \R's
functionality and supports plugins.

The third group of users, those who only require \proglang{R} for
certain tasks and do not wish to learn the language, are targeted by
task-specific GUIs. These interfaces usually do not contain a command
line, as the limited scope of the task does not require it. If a
task-specific GUI fits a task particularly well, it may even appeal to
an experienced user. There are many examples of task-specific GUIs in
\proglang{R}. Many GUIs assist in exploratory data analysis, including
\pkg{exploRase}, \pkg{limmaGUI},
\pkg{playwith}, \pkg{latticist}, and Rattle.  Other
GUIs are aimed at teaching statistics, e.g.,
\pkg{teachingDemos}. There are a few tools to automatically generate a
GUI that invokes a particular R function, such as the \pkg{fgui}
package and the \function{guiDlgFunction} function from the
\pkg{svDialogs} package.

All of these examples are within the scope of this book. We set out to
show that, for many purposes, adding a graphical interface to one's
work is not terribly sophisticated nor time-consuming.  This book does
not attempt to cover the development of GUIs that require knowledge of
another programming language, although several such projects
exist. One example is programming a Java/Swing GUI through
\pkg{rJava}, a native interface between \R\/ and Java. It is also
possible to extend the \pkg{RKWard} GUI using a mixture of XML and
Javascript, and the \pkg{biocep} GUI supports Java extensions. Our
focus is instead on programming GUIs with the \proglang{R} language.

The bulk of this text covers four different packages for writing GUIs
in \R. The \pkg{gWidgets} package is covered first. This provides a
common programming interface over several \R\/ packages that implement
low-level, native interfaces to GUI toolkits. The \pkg{gWidgets}
interface is much simpler -- and less powerful -- than the native
toolkits, so is useful for a programmer who does not wish to invest
too much time into perfecting a GUI. There are a few other packages
that provide a high-level \R\/ interface to a toolkit such as
\code{rpanel} or \code{svDialogs}, but we focus on \pkg{gWidgets}, as
it is the most general.

% ML: but gWidgets is much more general than either of those, right?
% JV: Yes

The next three parts introduce the native interfaces upon which
\pkg{gWidgets} is built. These offer fuller and more direct control of
the underlying toolkit and thus are well suited the development of GUIs
that require special features or performance characteristics.  The
first of these is the \pkg{RGtk2} package which provides a link
between \R\/ and the cross-platform \GTK\/ library. \GTK\/ is mature,
feature rich and leveraged by several widely used
projects. 

Another mature and feature-rich toolkit is \Qt, an open-source C++ library
from Nokia. The \R\/ package \pkg{qtbase} provides a native interface
from \R\/ to \Qt.  As \Qt\/ is implemented in C++, it is designed
around the ability to create classes that extend the \Qt\/
classes. \pkg{qtbase} supports this from within \R\/, although such
object oriented concepts may be unfamiliar to many \R\/ users.

Finally, we discuss the \pkg{tcltk} package, which interfaces with the
\TK\/ libraries. Although not as modern as \GTK\/ nor \Qt, these
libraries come pre-installed with the Windows binary, thus avoiding
installation issues for the average end-user. The bindings to \TK\/
were the first ones to appear for \R\/ and most of the GUI projects
above, notably \pkg{Rcmdr}, use this toolkit.

These four main parts are preceded by an introductory chapter on GUIs.

This text is written with the belief that much can be learned by
studying examples. There are examples woven through the primary text,
as well as stand-alone demonstrations of simple yet reasonably
complete applications. The scope of this text is limited to features
that are of most interest to statisticians aiming to provide a
practical interface to functionality implemented in R. Thus, not every
dusty corner of the toolkits will be covered. For the \pkg{tcltk},
\pkg{RGtk2} and \pkg{qtbase} packages, the underlying toolkits have
well documented APIs.

The package \pkg{\PACKAGENAME} accompanies this text. It includes the
complete code for all the examples. In order to save space, some
examples in the text have code that is not shown. The package provides
the functions \code{browsegWidgetsFiles}, \code{browseRGtk2Files},
\code{browseQtFiles} and \code{browseTclTkFiles} for browsing the
examples from the respective chapters. 


The authors would like to thanks the following people for their 
helpful comments made regarding draft versions of this book; Richie
Cotton, Erich Neuwirth, Jason Crowley, and Tengfei Yin.



