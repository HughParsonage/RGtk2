%% Web based GUIs -- Rpad, RApache, Rwui, ..



The internet affords one the opportunity to distribute their work in a convenient,
standardized way that allows people from around the globe to
share. Indeed, the \R\/ project has benefited greatly from the  web technologies
that enable user participation from disparate points.

This chapter shows some of the means to produce interactive interfaces
between the user and R through web technologies, at the time of
writing. Web interfaces to expose some resource have many obvious advantages
over the desktop interfaces discussed in previous chapters: no
installation issues for \R\/ and the toolkit libraries, user
familiarity with the browser interface, operating system independence, etc. This
makes it much easier to share ones work, but also puts an added burden
on the GUI writer, who must have some familiarity with new
technologies and the security implications contained therein. 

The web programmer coming to \R\/ will find relatively simple tools as
compared say to some open-source tools available for the python
programmer (Django \url{djangoproject.com}, pyjamas \url{pyjs.org},
...) or the ruby programmer (Ruby on Rails \url{rubyonrails.org}) or
even the web programmer used to one of the many available frameworks
built on PHP (Drupal \url{drupal.org}, Joomla! \url{joomla.org},
...). However, we will see that there are useful tools for \R\/ that
make it possible to develop \R-driven websites. Of course, web
technologies are changing quite rapidly, and \R\/ package writers are
hard at work, so one should check to see if newer, more powerful
resources, have been added to the mix.

This chapter does not even pretend to be comprehensive. It covers an
enormous array of technologies. Rather, its focus is to show how \R\/
can be used with these technologies. The interested reader will likely
need to seek additional help before implementation.


\section{Authoring Web Pages}
\label{sec:authoring-web-pages}
%% The request process

% COMMENT oN XHTML

% Form w3.org
% Uniform Resource Identifier (URI)

% A Uniform Resource Identifier (URI) is a string of characters which identifies an Internet Resource.

% The most common URI is the Uniform Resource Locator (URL) which identifies an Internet domain address. Another, not so common type of URI is the Universal Resource Name (URN).




The simplest web page is a static page that is returned when a user
makes a request. The basic architecture involves a browser (or some
other client) requesting a document from a web server. The request
must encode what document is desired so the web server can find
it. The request is specified in terms of a \defn{URI}, or uniform
resource identifier (a URL is technically a type
of URI). The web server in turn maps the URI request to a file on the
file system which the web server returns to the browser.

\begin{figure}
  \centering
\begin{verbatim}
browser -> request -> server -> page lookup -> return page to browser -> display
XXX -- REPLACE ME -- XXX
\end{verbatim}
  \caption{Basic flow of a how a static HTML file is displayed on a browser.}
  \label{fig:static-html-file}
\end{figure}


The type of HTML file just described is known as a static file, in contrast to a
dynamically generated file, as its contents do not reflect any
possible extra information in the request. The authoring of static HTML files
may involve three different techonologies described next.


\subsection{Markup languages}
\label{sec:markup-languages}

Typically a static web page is marked up in HTML. This now familiar
markup language allows the page author to indicate structure in
various parts of the document. Typical structures are paragraphs,
headers, images, etc. Additionally, markup can denote presentation, such
as color, font etc.

HTML is centered around the concept of a \dfn{tag} which is used to wrap a
portion of the text of a file. A tag has a name or keyword, in lower
case, and is enclosed in angle brackets. If the tag encloses some
text, it has a start and end style. The start tag for a tag \code{x}
would be \verb+<x>+ where the end tag would be \texttt{</x>} (an extra
slash). All text between theses tags would carry this tag. Some tags,
such as the image tag \tagger{img}, are used to define their attributes
only (a url of the file in this case) so do not come in pairs, in this case it is
common practice to end the tag with \texttt{/>}.~\footnote{There are
  two common variants of HTML one coming from SGML, another, XHTML, deriving
  from XML. Both are similar, but xhtml is stricter with its use of
  tags. Some basic rules (as opposed to conventions) include all tags
  are either ended with a closing tag, or with \texttt{/>}; tags are
  lower case; attributes must be enclosed in quotes and specified; the
  root element is different from that given in the examples. The
  Web Hypertext Application Technology Working Group
  (\url{http://whatwg.org}) has proposed specifications for the two
  that seem likely to become the standard for HTML5.}

A few typical tags are specified in
Table~\ref{tab:HTML-tags}.~\footnote{The site
  \url{http://www.w3schools.com/} provides a comprehensive, yet
  accessible, listing/}
Tags may indicate how text is supposed to be formatted (e.g. \texttt{b}),
others indicate what type of text it is (e.g. \texttt{code}), others
the document structure (e.g., \texttt{h1}, \texttt{p}, etc.).

\begin{table}
\centering
\label{tab:HTML-tags}
\caption{Table of common tags in HTML.}
\begin{tabular}{@{}lp{0.7\textwidth}@{}}
\toprule

Tag&Description\\
\midrule
\code{html}&Denotes an HTML file\\\code{head}&Marks header of file\\\code{body}&Marks off main body of file\\\code{script}&Used to include other types of files\\\code{p}&A paragraph. Also, \code{br} for a line break\\\code{h1}&First level header. Also \code{h2},...,\code{h6}\\\code{ul}&Unordered list. Also \code{ol}\\\code{li}&Denotes a list item\\\code{a}&An anchor for a hyperreference\\\code{img}&Denotes an image\\\code{div}&A text division, indicates a line break\\\code{span}&A text division, no implied break\\\code{b}&Denotes text to be set in bold\\\code{code}&Denotes text that is code\\\code{em}&Denotes text to be emphasized\\\code{table}&Creates a table element
\\ \bottomrule
\end{tabular}
\end{table}
A tag may have one or more \defn{attributes} specified. For example,
the anchor tag, \texttt{a}, has an attribute \tagattr{href}{a} to
specify the link that will open withn the user clicks on the
anchor. This attribute is indicated by name with an equals
sign. Quotes are optional for HTML, but recommended in general. They
are mandatory if there is white space involved. An example might be
\verb+<a href="http://www.r-project.org" />+.

All tags may have an \texttt{id} attribute specified, which is used to
give a unique ID to the part enclosed by the tag. This is used to
identify the tag within the document object model (DOM) described in
brief later. All tags may also have a \texttt{class} attribute to
indicate if the tagged content should be treated as a member of a
class. This provides a means to classify and treat similar objects as
a group. Some tags also allow one to specify style attributes, but a
more modern approach is to use a stylesheet to specify those. The
\texttt{span} and \texttt{div} tags are primarily used to specify
attributes for the tagged text.
\\

Some characters, such as angle brackets, have a reserved meaning.  As
such, to use an angle bracket in an HTML document requires the use an
\dfn{HTML entity reference}.  There are many such entities -- they are
also used for cahracter encodeings. Entities are
denoted by a leading ampersand \texttt{\&} and trailing semicolon, as
with \code{\&lt;} pr \code{\&gt;}. 


\begin{example}{Simple HTML file}{eg:sample-html-xhtml-header}
  A basic HTML file would include a structure similar to the following
  which shows the head and body. Within the head, a title is set.
  \begin{HTMLinput}
<!DOCTYPE HTML PUBLIC "-//IETF//DTD HTML//EN">
<html>
  <head>
    <title>Hello World Page</title>
  </head>
  <body>
    Hello world
  </body>
</html>
\end{HTMLinput}

A basic \code{xhtml} file has a different header, but otherwise
appears similar. For example the
following which specifies a version for the XML and a default name
space through the \tagattr{xmlns}{html} attribute.
\begin{HTMLinput}
<?xml version="1.0" ?>
<html xmlns="http://www.w3.org/1999/xhtml" xml:lang="en" 
  lang="en">
<head>
<meta http-equiv="content-type" 
  content="text/html; charset=UTF-8"/>
<title>Page title</title>
</head>
\end{HTMLinput}

\end{example}

\begin{example}{A basic table}{}
  Displaying tables is a common task for web pages. The \tagger{table}
  tag encloses a table. New rows are enclosed in a \tagger{tr} tag,
  and each cell can be a header cell, \tagger{th}, or a data cell
  \tagger{td}. The following shows one way alternate rows can be
  striped by hard coding a background color attribute (\code{bgcolor})
  to the rows.
  \begin{HTMLinput}
<table border="0" cellpadding="0">
  <tr>
    <th>Header 1</th><th>Header 2</th>
  </tr>
  <tr>
    <th>1</th> <th>2</th>
  </tr>
  <tr bgcolor="goldenrod">
    <th>3</th> <th>4</th>
  </tr>
</table>
 \end{HTMLinput}
\end{example}

\begin{example}{\R\/ helpers}{}
  Writing a tag and specifying a table can be tedious. Some helper
  functions are useful. The \pkg{hwriter} package includes a few, for
  now we mention \function{hmakeTag} which will produce a tag around
  some specified content along with attributes, that can be passed in
  through \R's \code{name=value} syntax. The first argument specifies
  the tag, and the second the values to be wrapped within the tag.
\begin{Schunk}
\begin{Sinput}
 require(hwriter)
 out <- hmakeTag("td",1:2, bgcolor="red")
 cat(out, sep="\n")
\end{Sinput}
\begin{Soutput}
<td bgcolor="red">1</td>
<td bgcolor="red">2</td>
\end{Soutput}
\end{Schunk}
The function is vectorized, as can be gathered from the output.
\end{example}


\subsection{Style sheets}
\label{sec:style-sheets}

%% http://www.w3.org/Style/LieBos2e/history/ for history

Casscading Style Sheets (\defn{CSS}) may be used to specify the
presentation of the text on a page. Common practice is to use the
markup language to specify document structure and a separate style sheet
to specify the layout of the first document. The advantage is a clean
separation of tasks so that one can make changes to the layout, say,
without affecting the text (and vice versa). A typical usage is to be
able to provide different layouts depending on the type of device.

Without going into detail, the style sheet syntax provides a means to
specify what type of tagged content the style will apply to (the
selector) and a means to specify what styles of markup will be
applied. For example, the specification below has
\code{h1, h2, h3, h4, h5, h6} as a \dfn{selector} to indicate that it
applies to all header tags. In the \dfn{declaration block} are style
specifications for the color of the text and the font
weight. Additionally, specifications for margins and padding are
given, along with a border on the bottom around the element.
\begin{HTMLinput}
  h1, h2, h3, h4, h5, h6 {
    color: Black;
    font-weight: normal;
    margin: 0;
    padding-top: 0.5em;
    border-bottom: 1px solid #aaaaaa;
}
\end{HTMLinput}
The full specification allows for much more complicated selections, be
they based on id of the tag (indicated with a prefix \code{\#}), class
of the tag (indicatd with a prefix \code{.}), or relation of tag to an
enclosing tag (left to right).  Style sheets can also refer to
positioning of the object within the page. Most modern web pages use
style sheets for layout, rather than tables, as it allows for greater
accessibility and offers advantages with search engines.


\begin{example}{A striped table using style sheets}{}
  Using the \tagattr{bgcolor}{table}  attribute of a table is deprecated in favor of
  style sheets for good reason. Here we illustrate a style sheet
  approach to striping a table. The style sheet may be defined in the
  HTML file itself with the \tagger{style} tag that appears within the
  document's \tagger{head}. 
  
  \begin{HTMLinput}
<style type="text/css">
table { border: 1px solid #8897be; border-spacing: 0px}
tr.head {background-color:#ababab;}
tr.even { background-color: #eeeeee;}
tr.odd { background-color: #ffffff;}
</style>
 \end{HTMLinput}
 A more common alternative, is to use the \tagger{link} tag to include
 the stylesheet through a url. For example,
\begin{HTMLinput}
<link rel="stylesheet" href="the.url.of.the.sheet" />
\end{HTMLinput}

 For the table itself, we need only replace the specification of the attribute with a
 class specification.
 \begin{HTMLinput}
<table>
  <tr class="head">
    <th>Header 1</th><th>Header 2</th>
  </tr>
  <tr class="odd">
    <th>1</th> <th>2</th>
  </tr>
  <tr class="even">
    <th>3</th> <th>4</th>
  </tr>
</table>
\end{HTMLinput}
  
\end{example}


\subsection{JavaScript}
\label{sec:javascript}

The third primary component of most modern web pages is JavaScript. This is a scripting
language that runs within the browser that allows manipulation of the
document. The document object model (DOM) specifies the elements of
the text that may be referenced from within JavaScript. For example,
individual elements can be found by unique ID, or common elements by class, or
elements sharing a tag, say \tagger{p}. JavaScript provides methods for
manipulating these elements. The simplest uses might be to change the
text when the mouse hovers over an element.

JavaScript allows web pages to be dynamic interfaces. The language
allows for callbacks to be defined for certain events, as with the
other GUI toolkits we've discussed. We don't pursue this, but note
that the \pkg{gWidgetsWWW} package uses JavaScript to make dynamic web
pages (cf. Figure~\ref{fig:gWidgets-three-oses}).

\begin{example}{Simple use of JavaScript to make a button have an action}{eg:javascript-action-button}
  The \tagger{button} tag produces a visual button. This tag has several
  event attributes, including \tagattr{onmouseover}{button} and
  \tagattr{onclick}{button}. When these occur, the specified
  JavaScript code is called. Here we show how to change the documents background
  color on a mouseover, and how to display a message on a mouse click.
  \begin{HTMLinput}
<button
  onMouseOver="document.bgColor='red'; return true;"
  onMouseOut="document.bgColor="; return true;"
  onClick="alert('clicked button'); return true;" >
  Click me...
</button>
 \end{HTMLinput}
\end{example}

There are several open source JavaScript libraries available that
offer convenient interfaces to JavaScript and UI widgets. We
mention ExtJS (\url{www.extjs.com}), jQuery (\url{jQuery.com}), YUI
(\url{developer.yahoo.com/yoi}) and Dojo
(\url{www.dojotoolkit.org}). 


\subsection{\R\/ tools to assist with authoring web pages}
\label{sec:r-tools-authoring}

There are quite a few packages for \R\/ to faciliate the authoring web
of pages from within
\R. We mention a couple.

\subsubsection{The hwriter package}
\label{sec:hwriter-package}


The \pkg{hwriter} package simplifies the task of creating HTML tables
for \R\/ objects, such as a matrix or vector. The package has a
self-generated example page in HTML which is created by its
\function{showExample} function (Or \code{example(hwriter)}). The main
function \function{hwrite} maps \R\/ objects into table objects (by
default) and has many options to modify the attributes involved. By
default, it writes its output to a file. The helper function
\function{openPage} takes a file name and returns a text
connection. The \function{closePage} function will close it. In the
examples below, so as the output will print, we use the
\function{stdout} function instead for the connection.

The package's examples show many different usages, we illustrate a few below.


A hyperlink can be generated through the \argument{link}{hwriter} argument.
\begin{Schunk}
\begin{Sinput}
 hwrite("R project", link="http://www.r-project.org",
        page=stdout())
\end{Sinput}
\begin{Soutput}
<a href="http://www.r-project.org">R project</a>
\end{Soutput}
\end{Schunk}
Although this usage doesn't save typing, a vectorized call could
easily do so.


To create a simple table, we need only call the constructor on a matrix or
data.frame object:
\begin{Schunk}
\begin{Sinput}
 m <- matrix(1:4, ncol=2)
 hwrite(m, page=stdout())
\end{Sinput}
\begin{Soutput}
<table border="1">
<tr>
<td>1</td><td>3</td></tr>
<tr>
<td>2</td><td>4</td></tr>
</table>
\end{Soutput}
\end{Schunk}


To get alternate rows to be striped we could have the following:
\begin{Schunk}
\begin{Sinput}
 styles <- c("odd","even")
 hwrite(m, page=stdout(), row.class=rep(styles, length=nrow(m)))
\end{Sinput}
\begin{Soutput}
<table border="1">
<tr>
<td class="odd">1</td><td class="odd">3</td></tr>
<tr>
<td class="even">2</td><td class="even">4</td></tr>
</table>
\end{Soutput}
\end{Schunk}
The \code{row.class} value is recycled for each entry in the row.


\subsubsection{The \pkg{R2HTML} package}
\label{sec:pkgr2html-package}

The \pkg{R2HTML} provides the generic function \code{HTML} for
creating HTML output from \R\/ objects based on their class.  As with
\function{hwrite}, this function writes its output to a connection for ease of
generating a file.

As \function{HTML} is a generic function, its usage is straightforward. For a
numeric vector we have:
\begin{Schunk}
\begin{Sinput}
 library(R2HTML)
 HTML(1:4, file=stdout())
\end{Sinput}
\begin{Soutput}
<p class='integer'>1&nbsp; 2&nbsp; 3&nbsp; 4</p>
\end{Soutput}
\end{Schunk}
The class is written using the \tagattr{class}{} attribute, so a style
sheet can be used:
\begin{Schunk}
\begin{Sinput}
 HTML(c(TRUE, FALSE), file=stdout())
\end{Sinput}
\begin{Soutput}
<p class='logical'>TRUE&nbsp; FALSE</p>
\end{Soutput}
\end{Schunk}

Functions may be formatted:
\begin{Schunk}
\begin{Sinput}
 HTML(mean, file=stdout())
\end{Sinput}
\begin{Soutput}
<br><xmp class=function>function (x, ...) 
UseMethod("mean")
<environment: namespace:base></xmp><br>
\end{Soutput}
\end{Schunk}

For more complicated objects, such as matrices and data frames, the
\function{HTML} function has other arguments. For example, a border
and inner border can be set (we omit the output).
\begin{Schunk}
\begin{Sinput}
 HTML(iris[1:3,1:2], Border=10, innerBorder=5, file=stdout())
\end{Sinput}
\end{Schunk}

The package also includes a number of functions to facilitate the
drafting of HTML files within \R, including \function{HTMLInitFile},
\function{HTMLCSS}, \function{HTMLInsertGraph} and
\function{HTMLEndFile}.


\subsubsection{The \pkg{brew} package}
\label{sec:pkgbrew-package}

\R\/ has the wonderful facility \pkg{Sweave} that passes through a
\LaTeX\/ file and can replace \R\/ code with the code and output
generated by evaluating the code. The \pkg{R2HTML} provides a means to
do the same with HTML files. Whereas, the \pkg{ascii} package provides
a means to do so for several ascii-based syntaxes for markup, many of
which have tools to create HTML pages.

The \pkg{brew} package does something similar, yet different. It
allows one to place a template within an HTML file that \R\/ will
eventually populate when called accordingly. In the next section, we
illustrate how this can be used to produce dynamically generated web pages. For now, we
mention how to make a template and how to process it.


A template is a file with parts of it marked by delimiters
(cf. Table~\ref{tab:brew-delimiters}). All text not within delimiters
is processed as is. Whereas, text within delimiters may be evaluated
by \R, and if evaluated the contents may be inserted into the output
or simply used to adjust the evaluation environment. When processed
with brew, the result may be stored in a file, or sent to
\code{stdout}.


\begin{table}
  \centering
  \begin{tabular}{lr@{\quad}c@{\quad}c}
    \toprule
    &&\multicolumn{2}{c}{Evaluate}\\
    && Yes & No \\
    \cmidrule{3-4}
   Print & Yes & \verb+<%= %>+ & no delimiters\\
%   \addlinespace[.5pt]\\
          & No  & \verb+<%  %>+  & \verb+<%# %>+\\
\bottomrule
 \end{tabular}
  \caption{The \pkg{brew} delimiters and how they are processed.}
 \label{tab:brew-delimiters}
\end{table}


\begin{example}{Differences in \pkg{brew} delimiters}{eg:brew-delimiters}

To illustrate the differences in the \pkg{brew} delimiters, the left
side has \pkg{brew} commands and the right side is their output.

\begin{minipage}{0.45\linewidth}
  \HTMLinputlisting{brew-basic.brew}
\end{minipage}
%%
\quad\quad
\begin{minipage}{0.45\linewidth}
  \HTMLinputlisting{brew-basic.brew.out}
\end{minipage}
\end{example}

\begin{example}{Dynamically formatted text}{eg:brew-dynamic-text}
  This example shows how brew can be used to insert dynamic text.

This template

\HTMLinputlisting{brew-fortunes.brew}

produces

\HTMLinputlisting{brew-fortunes.brew.out}

\end{example}

\begin{example}{Recursively calling \function{brew}}{eg:recursive-brew}
  Typically there will be more than one page on a web site with each
  sharing common features: a banner, a footer, navigation links, a
  side bar, ... Using templates for these pieces and then including
  the template in a file is one way to centralize these common
  pieces. The \function{brew} function can easily be used to do this.
  
  For example, here we define a header and footer and then call them
  in from a page. Our header is basic template, but includes a
  variable \code{title} to be defined in the page.
  
\HTMLinputlisting{brew-header.brew}  

  Our basic footer is
  
\HTMLinputlisting{brew-footer.brew}    

  And a typical page has this structure. We set the variable
  \code{title} in the scope of this page, but it is seen within the
  scope of the call to process the header page.
  
\HTMLinputlisting{brew-page.brew}    
  
  
\end{example}


\begin{example}{Creating a template within a template}{ex:brew-template-withing}
  This example shows how one can define a template within a template,
  as an alternative to a separate file. The basic idea is to use
  \function{paste} to bypass the issue of being unable to nest
  \pkg{brew} delimiters. We evaluate the template within a context, so
  that each time we get the values from different rows.

This template
 \HTMLinputlisting{brew-list.brew}

 produces
 
 \HTMLinputlisting{brew-list.brew.out}
\end{example}

\subsection{Graphics in web pages}
\label{sec:graphics-web-pages}
Web pages may be plain text, but most contain images or graphics. The
\tagger{img} tag allows one to display a graphics file in an HTML
page by specifying its \tagattr{src}{img} attribute. This is an image file,
often in \code{png}, \code{gif} or \code{jpeg} format. In this
section, we describe how \R\/ can be used to generate images by using
different device drivers. To list all the possible stock devices, see the
help page for \code{Devices}. The function \function{capabilities} lists
which devices are available for a given \R\/ installation.

\subsubsection{png}
\label{sec:images}
Typically when a plot command is issued, an interactive plot device is
opened or reused, however, the user can specify a device to save the
output to a file for further use. For example, the \function{pdf} and
\function{postscript} functions will turn \R\/ commands into files for
inclusion in written documents. For web pages, the \function{png} and
\function{jpeg} device drivers are available for many systems. These
may be used to insert a graphic into a web page.

The basic usage is like that of the \function{pdf} driver illustrated
below -- open the device, issue graphics commands, close the device:
\begin{Schunk}
\begin{Sinput}
 pdf(file="test.pdf", width=6, height=6) # in inches
 hist(rnorm(100), main="Some graphic")
 invisible(dev.off())                    # close device
\end{Sinput}
\end{Schunk}

To use the \code{png} driver on a linux server, the option \code{type}
should be set to \code{cairo} either through the constructor, or by
setting the option \code{bitmapType}. 

The \pkg{Cairo} device driver is an alternative which can also output
in png format.


\subsubsection{SVG graphics}
\label{sec:svg-graphics}
The web has other means to display graphics than an inclusion of an
image file. For example, Flash is a very popular method.~\footnote{The
  \pkg{FlashMXML} from \url{omegahat.org} provides a means to
  genearate flash files from within \R.} SVG (Scalable vector graphics)~\footnote{\url{http://www.w3.org/Graphics/SVG/}} is another way to specify graphical objects using XML. Many modern web browsers have support for
the display of SVG graphics. To insert the file, we have the
\tagger{object} tag and its attributes \tagattr{data}{object} and
\tagattr{type}{object}, as in
\begin{HTMLinput}
<object data="image-svg.svg" type="image/svg+xml"></object> 
\end{HTMLinput}
Not all browsers support svg, so one might also have a fall back
image, as in:
\begin{HTMLinput}
<object data="image-svg.svg" type="image/svg+xml">
<img src="image-png.png" alt="alternative file" /> 
</object> 
\end{HTMLinput}

There are a few drivers to create SVG files in \R, for example In the base
\pkg{grGraphics} package, the driver \code{svg} is available.  This non-interactive
driver is used as the \code{png} one illustrated above.


The \pkg{RSVGTipsDevice} package provides an alternate driver,
\function{devSVGTips}.  The ``Tips'' part of the package, is provided
by the function \function{setSVGShapeToolTip}, which allows one to
specify a tooltip to popup when the mouse hovers over an element. The
tooltip specified is placed over the next shape drawn, such as a
point.

For example, here we add a tip and a URL to each point in a
scatterplot. We initially call \function{plot} without plot characters to
set up the axes, etc.
\begin{Schunk}
\begin{Sinput}
 require(RSVGTipsDevice)