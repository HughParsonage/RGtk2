%%% These go inside begin document, so in a different file


%% From http://tolstoy.newcastle.edu.au/R/e6/help/09/04/11790.html
\newenvironment{Schunk}{}{}
%%\usepackage{listings}
\definecolor{gris05}{gray}{0.05}
\definecolor{gris10}{gray}{0.10}
\definecolor{gris40}{gray}{0.40}
\definecolor{gris90}{gray}{0.90}
\definecolor{gris95}{gray}{0.95}
\lstloadlanguages{R}

\lstnewenvironment{Sinput}[1][]{
  \lstset{
    language={R},
    basicstyle=\small,                        % print whole listing small
    %
    % keywordstyle=\color{black}\bfseries,      % style for keyword
    keywordstyle=\ttfamily,      % style for keyword
    %%
    %
    emph={in},                                % define a list of word to emphasis
    % emphstyle=\color{black}\bfseries,         % define the way to    emphase
    emphstyle=\ttfamily,         % define the way to emphase
    % emph={[2]out},                            % define a second list of word to emphasis
    % emphstyle=[2]\color{red}\bfseries,        % define the way to emphase the list 2
    %
    commentstyle=\color{gris40}\small\itshape,%\sffamily, % comments
    %
    stringstyle=\ttfamily,
    identifierstyle=\ttfamily,
    %% 
    frame=leftline,                             % box arround the code
    backgroundcolor=\color{gris95},           % background color
    showspaces=false,                         % show the space in code, or not
    stringstyle=\color{gris05}\ttfamily,                    % style of the string (like "hello word")
    showstringspaces=false,                   % show the space in
                                % string, on not
    aboveskip=\medskipamount,
 #1
  }
}{}

\lstnewenvironment{Soutput}[1][]{
  \lstset{%
    %% No language here, otherwise emphasis is odd
    %% language={R},                           
   basicstyle=\ttfamily\small,                        % print whole listing small
%%    keywordstyle=\color{black}\bfseries,      % style for keyword
    %% emph={in},                                % define a list of word to emphasis
    %% emphstyle=\color{black}\bfseries,         % define the way to emphase
    % emph={[2]out},                            % define a second list of word to emphasis
    % emphstyle=[2]\color{red}\bfseries,        % define the way to emphase the list 2
    %% frame=single,                             % box arround the code
    backgroundcolor=\color{gris95},           % background color
    %% commentstyle=\color{gris10},              % define the style of the comments
    stringstyle=\ttfamily,                    % style of the string (like "hello word")
    showspaces=false,                         % show the space in code, or not
    showstringspaces=false,                   % show the space in  string, on not
   #1
  }
}{} 




% %% Override the verbatim environment for Sinput, Soutput, Scode

% \DefineVerbatimEnvironment{Sinput}{Verbatim}{
% fontfamily=lmss,
%   fontseries=b
%   ,fontsize=\relsize{0}
% %%  ,xleftmargin=2em
% }
% \DefineVerbatimEnvironment{Soutput}{Verbatim}{
%   formatcom=\color{blue},
% %%  fontshape=sl,
%  fontsize=\relsize{0},
%   showspaces=FALSE
% %%  ,xleftmargin=2em
% %%  formatcom=\color{gray40}
% }
% \DefineVerbatimEnvironment{Scode}{Verbatim}{
%   %% xleftmargin=2em
%   %% fontshape=sl
% }

% %% Same as Sinput
% \DefineVerbatimEnvironment{CodeSnippet}{Verbatim}{
% %  formatcom=\color{RInput},
%   fontseries=b, 
%   fontsize=\relsize{-1}
% }


% %%% DUncan Murdoch's solution to tighten up space in Sweave
% \fvset{listparameters={\setlength{\topsep}{0pt}}}
% \renewenvironment{Schunk}{\vspace{\topsep}}{\vspace{\topsep}}

% % %% use   fontshape=it, for italics
% % \DefineVerbatimEnvironment{Sinput}{Verbatim}{
% % %  formatcom=\color{RInput},
% %   fontfamily=courier,
% %   fontseries=b,
% %   fontsize=\relsize{-1},
% %   showspaces=false
% % %%  baselinestretch=0
% % }
% % \DefineVerbatimEnvironment{Soutput}{Verbatim}{
% % %  formatcom=\color{ROutput},
% %   fontfamily=courier,
% %   fontseries=b,
% %   fontsize=\relsize{-2},
% %   showspaces=false
% % %%  baselinestretch=0
% % }


%%% An environment for HTML text
\lstnewenvironment{HTMLinput}[1][]{
  \lstset{%
    language={html},
    basicstyle=\small,                        % print whole listing small
    % keywordstyle=\color{black}\bfseries,      % style for keyword
    keywordstyle=\ttfamily,      % style for keyword
    emph={in},                                % define a list of word to emphasis
    % emphstyle=\color{black}\bfseries,         % define the way to    emphase
    emphstyle=\ttfamily,         % define the way to emphase
    % emph={[2]out},                            % define a second list of word to emphasis
    % emphstyle=[2]\color{red}\bfseries,        % define the way to emphase the list 2
    % commentstyle=\color{gris10}\ttfamily,              % define the
                                % style of the comments
    stringstyle=\ttfamily,
    identifierstyle=\ttfamily,
    %% 
    frame=leftline,                             % box arround the code
    backgroundcolor=\color{gris95},           % background color
    showspaces=false,                         % show the space in code, or not
    stringstyle=\ttfamily,                    % style of the string (like "hello word")
    showstringspaces=false,                   % show the space in
    % string, on not
    aboveskip=\medskipamount,
 #1
  }
}{}

%% Use to call in an HTML file
\newcommand{\HTMLinputlisting}[1]{
  \lstinputlisting[
  language={html},
  basicstyle=\small\ttfamily,
   emph={in},                                % define a list of word to emphasis
   emphstyle=\ttfamily,         % define the way to emphase
   stringstyle=\ttfamily,
   identifierstyle=\ttfamily,
%   %% 
   frame=leftline,                             % box arround the code
   backgroundcolor=\color{gris95},           % background color
   showspaces=false,                         % show the space in code, or not
   stringstyle=\ttfamily,                    % style of the string (like "hello word")
   showstringspaces=false,                   % show the space in
%   % string, on not
  aboveskip=\medskipamount
  ]{#1}
}

% %%% Local Variables: 
% %%% mode: latex
% %%% TeX-master: t
% %%% End: 
