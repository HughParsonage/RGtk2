%% Setup file for latex
%%%%%%%%%%%%%%%%%%%%%%%%%%%%%%%%%%%%%%%%%%%%%%%%%%
%% Load packages
\usepackage{mathptmx}  %% 420 pages          % for math fonts type 1
\usepackage[pdftex]{graphicx}           % for graphics files
\usepackage{floatflt}           % for ``floating boxes''
\usepackage{index}
\usepackage{relsize}            % for relative size fonts
\usepackage{amsmath}            % for amslatex stuff
\usepackage{amsfonts}           % for amsfonts
\usepackage{url}                % for \url,
\usepackage{listings}
\usepackage{booktabs}

%%\usepackage{subfigure}          % for subcaption command
%%%%%%%%%%%%%%%%%%%%%%%%%%%%%%%%%%%%%%%%%%%%%%%%%%
%%% The page


%%
%% Set up memor for book layout
%% paragraph

%% showtrims
%%\showtrimson
%%\trimXmarks


%%%%%%%%%%%%%%%%%%%%%%%%%%%%%%%%%%%%%%%%%%%%%%%%%%

%%%%%%%%%%
%% page layout
%% assume basic page size for now
\raggedbottom


%%%%%%%%%%
%% fonts
%% Review:
% Upright shape \textup{Upright shape} 
% Italic shape \textit{Italic shape} 
% Slanted shape \textsl{Slanted shape} 
% S MAL L CAP S S HAP E \textsc{Small Caps shape} 
% Series or weight 
% Medium series \textmd{Medium series} 
% Bold series \textbf{Bold series} 
% Family 
% Roman family \textrm{Roman family} 
% Sans serif family \textsf{Sans serif family} 
% Typewriter family \texttt{Typewriter family} 

%% which fonts?
%%\usepackage{avant}
%%\usepackage{palatcm}
\usepackage[T1]{fontenc}
%\usepackage[adobe-utopia]{mathdesign}
%\usepackage{aurical}

\renewcommand{\encodingdefault}{T1}
\usepackage[sc]{mathpazo}
\linespread{1.05}         % Palatino needs more leading (space between lines)

%% verbatim fonts
%\setverbatimfont{\fontencoding{T1}\fontfamily{cmss}\selectfont} 
%\setverbatimfont{\normalfont\ttfamily}





%%%%%%%%%%
%% titles

%%%%%%%%%%
%% divisions

%% chapter styles
\chapterstyle{ell}
%\chapterstyle{memman}
%\chapterstyle{veelo}
%\chapterstyle{bringhurst}
% \chapterstyle{crosshead}
% \chapterstyle{dowding}
% \chapterstyle{komalike}

% \chapterstyle{ntglike}
%\chapterstyle{tandh}
%\chapterstyle{wilsondob}

%%% pagestyle
\pagestyle{ruled}
%\pagestyle{companion}

% captions -- The class uses the following to specify the standard LaTeX caption style: 
% \captionnamefont{} 
% \captiontitlefont{} 
\captionstyle[\centering]{\raggedright} 
\captionwidth{\linewidth} 
% \normalcaptionwidth 
% \normalcaption 
\captiondelim{: } 
%\postcaption{\rule{\linewidth}{0.4pt}\par}

%%%%%%%%%%
%% pagination headers

%%%%%%%%%%
%% paragraphs, lists
\tightlists
%% WHAT TO DO HERE? \renewcommand{\paragraph}[1]{\marginpar{#1}}

%%%%%%%%%%
%% content lists

%%%%%%%%%%
%% floats and captions

%%%%%%%%%%
%% rows and columns

%%%%%%%%%%
%% page notes

%%%%%%%%%%
%% decorative text


%%%%%%%%%%
%% Boxes verbatims files

%%%%%%%%%%
%% cross referencing

%%%%%%%%%%
%% back matter





%%
%%
%%%%%%%%%%%%%%%%%%%%%%%%%%%%%%%%%%%%%%%%%%%%%%%%%%


%% http://article.gmane.org/gmane.comp.lang.r.general/137990
% \usepackage{listings}
% \lstset{%
% 	basicstyle=\scriptsize,
% 	breaklines=true,
% 	frame=single,
% 	literate=
% 		{<-}{$\leftarrow$}{2}
% }
% \renewcommand\lstlistlistingname{List of listings}

%% This breaks long commands by hyphenating. It beats the alternative
%% of having them run all over the margin
\usepackage{everysel}
\EverySelectfont{%
\fontdimen2\font=0.4em% interword space
\fontdimen3\font=0.2em% interword stretch
\fontdimen4\font=0.1em% interword shrink
\fontdimen7\font=0.1em% extra space
\hyphenchar\font=`\-% to allow hyphenation
}


%%%%%%%%%%%%%%%%%%%%%%%%%%%%%%%%%%%%%%%%%%%%%%%%%%
%% Abbreviations (most are Rd-ish)
%% Flag something to look at -- XXX is easy to search for
\newcommand{\XXX}[1]{}%%{XXX-- #1 --XXX\\}




\newcommand{\R}{\textsf{R}}
\newcommand{\code}[1]{\texttt{#1}} % code
\newcommand{\qcode}[1]{\code{"#1"}} % quoted code 
\newcommand{\Rcode}[1]{\code{#1}}
\newcommand{\proglang}[1]{\code{#1}} % a programming lang
\newcommand{\software}[1]{\code{#1}} % some software

\newcommand{\defn}[1]{\textit{#1}\index{\textit{#1}}}   % add in index
\newcommand{\command}[1]{\code{#1}} % name of command
\newcommand{\function}[1]{\code{#1}} % name of function
\newcommand{\Rfunction}[1]{\function{#1}} % name of function
\newcommand{\Robject}[1]{\code{#1}} % name of function
\newcommand{\constructor}[1]{\function{#1}\index{#1}}

\newcommand{\args}[1]{\code{#1}} % name of argument only
\newcommand{\argument}[2]{\args{#1}\index{#2|\texttt{#1}}} % name of an argument, plus
                                % function for index
\newcommand{\subcommand}[2]{\textit{#2} \args{#1}\index{#2|\code{#1}}} % name of an tk subcommand plus
                                % function for index
\newcommand{\subcommanda}[3]{\subcommand{#1}{#2} \textit{#3} }
\newcommand{\option}[2]{\args{#1}\index{#2|\code{#1}}} % name of an option plus constructor
\newcommand{\class}[1]{\code{#1}}  % a class
\newcommand{\Rclass}[1]{\class{#1}}  % an R class
\newcommand{\generic}[1]{\code{#1}} % name of generic method -- no
\newcommand{\meth}[1]{\generic{#1}}     % single arg, no class
\newcommand{\method}[2]{\meth{#1}\index{#2|\code{#1}}} % name of method with
                                % class for index
\newcommand{\prop}[1]{\generic{#1}}     % single arg, no class
\newcommand{\property}[2]{\prop{#1}\index{#2|\code{#1}}} % name of
                               % widget property
\newcommand{\qtproperty}[1]{\prop{#1}} % qt property
\newcommand{\qtenumeration}[1]{\generic{#1}} % an enumerated flag in
\newcommand{\enumeration}[1]{\qtenumeration{#1}}
                                % Qt$Qt object
\newcommand{\OR}{$\mid$}        % for |.QtEnum
\newcommand{\AND}{\&}        % for |.QtEnum
\newcommand{\signal}[1]{\code{#1}} % name of signal
\newcommand{\event}[1]{\code{#1}} % name of event
\newcommand{\dfn}[1]{\textit{#1}} % definition
\newcommand{\secdfn}[2]{\textit{#1}} % a definition with section
                                % (tcltk, gWidgets, ...) for index
\newcommand{\dfnref}[1]{\textit{#1}} % refer to a definition
\newcommand{\env}[1]{\texttt{#1}} % environment setting
\newcommand{\file}[1]{\texttt{#1}}
\newcommand{\kbd}[1]{\textmd{#1}}
\newcommand{\pkg}[1]{\texttt{#1}}
\newcommand{\opt}[1]{\texttt{#1}} % R option
\newcommand{\acronym}[1]{\texttt{#1}}

%% HTML
\newcommand{\tagger}[1]{\texttt{#1}}
\newcommand{\tagattr}[2]{\texttt{#1}} % #2 is the tag for index

\usepackage{fancyvrb}
\DefineShortVerb{\|}
\SaveVerb{ASSIGN}|<-|
%%\newcommand{\ASSIGN}{\code{\UseVerb{ASSIGN}}} % <- formats funny
\newcommand{\leftBracket}{$<$}
\newcommand{\rightBracket}{$>$}
%% THisis from the highlight pacakge
\newsavebox{\hlboxlessthan}%
\setbox\hlboxlessthan=\hbox{\verb.<-.}%
\newcommand{\ASSIGN}{{}{\usebox{\hlboxlessthan}}{}}
%%\newcommand{\ASSIGN}{\code{$<$-}}  %% <-
\newcommand{\backslashn}{\code{$\backslash$n}} %% \n

\newcommand{\GTK}{GTK+}
\newcommand{\TCL}{Tcl}
\newcommand{\Tcl}{\TCL}
\newcommand{\TK}{Tk}
\newcommand{\Tk}{Tk}
\newcommand{\tcltk}{Tcl/Tk}
\newcommand{\Qt}{Qt}
\newcommand{\wxWidgets}{wxWidgets}
\newcommand{\Java}{Java}
\newcommand{\gWidgets}{gWidgets}

\newcommand{\TITLE}{Programming GUIs within R}
\title{\TITLE}
\newcommand{\PACKAGENAME}{ProgGUIInR}
\newcommand{\WINDOZE}{Windows}
\newcommand{\UNIX}{Unix}
\newcommand{\LINUX}{Linux}
\newcommand{\OSX}{Mac OS X}
%%
%%%%%%%%%%%%%%%%%%%%%%%%%%%%%%%%%%%%%%%%%%%%%%%%%%

%% Penalties 
\widowpenalty=30000
\clubpenalty=30000


\usepackage{color}
%%% Define some colors
\definecolor{gray70}{gray}{.70}
\definecolor{gray60}{gray}{.60}
\definecolor{gray50}{gray}{.50}
\definecolor{gray40}{gray}{.40}
\definecolor{gray25}{gray}{.25}

%% \usepackage{fancyhdr} % in memoir



\usepackage{fancyvrb}
%\usepackage{makeidx}
\usepackage{multicol}          % for making multiple columns
%%\usepackage{fancybox}           % for framing

\usepackage{prelim2e}           % put on bottom of each page
\renewcommand{\PrelimWords}{Draft version, do not circulate}

%\usepackage{/home/verzani/R/lib/R/share/texmf/Sweave}
%\usepackage{Sweave}

\usepackage{lscape}             % landscape tables
%% Save space things
%% http://www-h.eng.cam.ac.uk/help/tpl/textprocessing/squeeze.html 
%% \usepackage{float}        
% \usepackage{jvfloatstyle}       % redefine float.sty for my style. Hack
% \floatstyle{jvstyle}            
% \restylefloat{table}
% \restylefloat{figure}


\usepackage{natbib}
\bibpunct{(}{)}{;}{a}{,}{,}

\newcommand{\warning}[1]{Warning #1}
\newcommand{\aside}[1]{Aside #1}


%% An environment to flag an example to insert in index, allow for
%% referencing, and allow formatting
%% two arguments: title, label

%% XXX Fix me to make look nicer
\newcounter{Example}[chapter]
\def\theExample {\thechapter.\arabic{Example}}
\setcounter{Example}{1}

\newenvironment{example}[2]{
  \refstepcounter{Example} 
  \vspace{12pt}
  \noindent
  \textbf{Example \theExample: #1}
  \label{#2}
  \newline
}{
  %% how to finish
}


% \newenvironment{example}[2]{
%   \refstepcounter{Example} 
%   {\vspace{12pt}\noindent#1\label{#2}\hrulefill\newline}}%
%   {\noindent\hrulefill}

% \newenvironment{example}[2]{
%   \refstepcounter{Example} 
%   {\vspace{18pt}\noindent\large\sf Example \theExample: #1 \hfill}\label{#2}
% }{
%   \noindent{\rule{2in}{1pt}\vspace{12pt}}
% %% something at the finish too?
% }
%% Numbered code example
%% wrap around <<>>= @ environements to get number and labeling possibility
\newcounter{CodeExample}[chapter]
\def\theCodeExample {\thechapter.\arabic{CodeExample}}
\setcounter{CodeExample}{1}

\newenvironment{codeexample}[2]{
  \refstepcounter{CodeExample} 
  \noindent{Example \theCodeExample}\newline
  \hrulefill \newline}{
  {\noindent\hrulefill}
}



%%%%%%%%%%%%%%%%%%%%%%%%%%%%%%%%%%%%%%%%%%%%%%%%%%
%% Listings
%% for code listings. Make pretty?
\newenvironment{listing}{\begin{figure}}{\end{figure}}
\newcommand{\definition}[1]{#1}

